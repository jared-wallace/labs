\documentclass[letterpaper,12pt]{article}
\usepackage[utf8]{inputenc}
\usepackage{fullpage}
\usepackage{courier}
\usepackage[margin=0.75in]{geometry}
\usepackage{listings}
\usepackage{color}
\usepackage{graphicx}
\usepackage[width=5in]{caption}
\usepackage{hyphenat}
\usepackage[section]{placeins}
\usepackage{cmll}

% Format a sectionless paragraph
\newcommand*\unparagraph{
	\par
	\nopagebreak
	\vskip3.25ex plus1ex minus.2ex
	\noindent
}

% define extra colors
\definecolor{dkgreen}{rgb}{0,0.6,0}
\definecolor{purple}{RGB}{159,0,197}

% define the code listing format
\lstset{
	language=C++,
	basicstyle=\footnotesize\ttfamily,
	backgroundcolor=\color{white},
	showspaces=false,
	showstringspaces=false,
	frame=none,
	tabsize=3,
	keywordstyle=\color{purple},
	commentstyle=\color{dkgreen},
	stringstyle=\color{blue},
	escapeinside={\%*}{*)}
}

% define the title/header
\title{\Large CS 1428 Honors\\Lab 2} 
\author{Jared Wallace}
\date{}

\begin{document}

\maketitle

\vspace{30mm}

\section*{Questions}

\begin{enumerate}
    \item (10 pts) Write a snippet of code that includes an if statement to compare the value of
        x (a variable inputted by the user) to the named constant \emph{NUM}, which you must declare
          and assign a value of your choice ($0<=NUM<=256$). You must prompt the user for the value of x.
    \vspace{40mm}
    \item (10 pts) Evaluate these logical expressions. Write the answers on this work sheet.
          Do NOT use the computer to evaluate these expressions.
          \begin{itemize}
              \item $T \with \with F$
            \item $T || F$
            \item $F \with \with F$
            \item $!( T \with \with T )$
            \item $!T \with \with T$
          \end{itemize}
    \vspace{30mm}
    \item (10 pts) What is the output for the following snippet of code?
        \begin{lstlisting}
        int main()
        {
            int x = 3;
            bool y = false;
            cout << x++ << endl;
            if(y && ++x == 5)
            {
                cout << "Hooray!"<<endl;
            }
            else
            {
                cout << "awww...."<<endl;
            }
            cout << x << endl;

            return 0;
        }
        \end{lstlisting}
    \item (60 pts) You will need to make a program named lab2h.cpp that will function as a
          basic calculator. Requirements:
        \begin{itemize}
            \item Declare the following constants (same as last week).
            \begin{itemize}
                \item OP\_ADD with a value of 0
                \item OP\_SUBTRACT with a value of 1
                \item OP\_MULTIPLY with a value of 2
                \item OP\_DIVIDE with a value of 3
                \item OP\_MOD with a value of 4
                \item OP\_EXPONENT with a value of 5
                \item OP\_READ with a value of 6
                \item OP\_WRITE with a value of 7
            \end{itemize}
            \item Declare the following variables:
                \begin{enumerate}
                    \item inst (integer)
                    \item data0 (integer)
                    \item data1 (integer)
                    \item data2 (integer)
                \end{enumerate}
            \item Prompt the user to input the value of inst, which will be the numerical
                  interpretation of the operation you wish to perform, followed by the values
                  of data1 and data2, which are the two numbers to perform the operation on.
            \item Use if statements to handle control flow. (You may use other tools if you know them)
            \item The results of the calculations should be stored in the variable data0.
            \item If the user selects an option that the calculator is not equipped to perform,
                  merely output the string “Unable to perform operation, assigning data0 to -1”
                  and assign data0 to -1.
            \item After each calculation, output a message to the user that their calculation is done
                  and output the result. Both messages should go to standard out.
            \item Your calculator must at least handle one operation per execution. You may, at your discretion,
                  code it to allow more than one.
        \end{itemize}

\end{enumerate}
\section*{Extra Credit}
(10 pts possible) What is the logical equivalent of the following statement and what law does it follow?
    (p and q represent two different values)

    $!( p \with \with q )$
\section*{Deliverables}
Hard copy of the source code you wrote (lab2h.cpp) and the answers to the questions.
Soft copy (upload to homework upload) of your source code.

% Comic at the bottom
\begin{figure}[ht!]
	\centering
	\includegraphics[width=5in]{academia_vs_business.png}
    \caption*{Some engineer out there has solved P=NP and it's locked up in an electric eggbeater calibration routine.  For every 0x5f375a86 we learn about, there are thousands we never see.}
\end{figure}
\end{document}
