\documentclass[letterpaper,12pt]{article}
\usepackage[utf8]{inputenc}
\usepackage{fullpage}
\usepackage{courier}
\usepackage[margin=0.75in]{geometry}
\usepackage{listings}
\usepackage{color}
\usepackage{graphicx}
\usepackage[width=5in]{caption}
\usepackage{hyphenat}
\usepackage[section]{placeins}
\usepackage{cmll}
\usepackage{float}
\usepackage{hyperref}

% Format a sectionless paragraph
\newcommand*\unparagraph{
	\par
	\nopagebreak
	\vskip3.25ex plus1ex minus.2ex
	\noindent
}

% define extra colors
\definecolor{dkgreen}{rgb}{0,0.6,0}
\definecolor{purple}{RGB}{159,0,197}

% define the code listing format
\lstset{
	language=C++,
	basicstyle=\footnotesize\ttfamily,
	backgroundcolor=\color{white},
	showspaces=false,
	showstringspaces=false,
	frame=none,
	tabsize=3,
	keywordstyle=\color{purple},
	commentstyle=\color{dkgreen},
	stringstyle=\color{blue},
	escapeinside={\%*}{*)}
}

% define the title/header
\title{\Large CS 1428 Honors\\Lab 8}
\author{Jared Wallace}
\date{}

\begin{document}

\maketitle

\vspace{30mm}

\section*{Overview}
Today we will be extending our assembler to make it Turing Complete. We are going to modify it so that
the assembler now incorporates a \emph{jump} instruction and a \emph{conditional jump} instruction.
Additionally, we will be \emph{refactoring} the assembler to utilize functions. Refactoring is the
process of modifying code so that the internal structure changes, but the external behavior remains
consistant. For our assembler, this will largely consist of simple cut and paste actions, but in the
real world it usually involves changing algorithms to be more efficient and other, more substantive
changes.
\section*{Questions}
\begin{enumerate}
    \item (15 pts) Last week, we had modified our assembler to read in the instructions for each program
        into "program" memory. When we did that, each instruction line gained an additional piece of
        associated data -- its \emph{index} in the program array. This index number functions as a line
        number does in a traditional program. Accordingly, just as you can specify a line number in C++ 
        to "goto", we wish to implement jumping to a certain index number for our assembler.

        If you recall, our assembler reads through the instruction lines by using a loop. We can
        therefore implement our "jump" instruction by manipulating the loops index variable.
        \emph{(Usually the index variable is 'i')}. Be careful! In a for loop, the index variable is updated,
        or incremented, when the for loop reaches the closing bracket. You must account for this in your
        jump function, lest you incur an "off by one" error. For now, write a simple case statement to
        correctly handle the jump command. Use the following format:
        \begin{lstlisting}
        8 [offset] 0 0
        \end{lstlisting}
        Hint: offset values can be negative to effect a jump "backwards"
    \vspace{30mm}
    \item (15 pts) Our conditional jump instruction will take the form of a "Branch if Equal"
        statement. In other words, the instruction will have three values: the offset and two index values
        of data to compare. If the two data values are equal, the jump occurs to the specified offset. Otherwise,
        nothing changes and execution continues at the next instruction. The instruction will take the following
        form:
        \begin{lstlisting}
        9 [offset] [index 1] [index 2]
        \end{lstlisting}
        Write the case statement implementing this feature.
    \vspace{40mm}
\item (70 pts) Using the answers from questions one through two, modify your
    assembler so that it implements both \emph{jump} and the \emph{conditional jump}.
    Your program should comply with the following requirements:
    \begin{itemize}
        \item Reads all instructions into memory at one time.
        \item Processes the instructions in order, one line at a time.
        \item Properly handles the same sample programs from last time (supplied)
        \item Properly handles one new program that includes jumps and conditional jumps
        \item Implements each type of instruction as its own function, as well as creating
            a function to handle the initial reading and storage of instructions into program
            memory. Use the following prototype for this:
            \begin{lstlisting}
            int loadProgram(string filename, int program_memory[512][4]);
            \end{lstlisting}
            The value of the integer returned should be the number of lines of instructions
            read into memory. If the operation fails, it should return 0.
    \end{itemize}
\end{enumerate}
\section*{Deliverables}
Hard copy of the source code you wrote (assembler.cpp). Soft copy (upload to homework upload) of
your source code. As always, make sure you make a commit and push your work.

% Comic at the bottom
\begin{figure}[ht!]
	\centering
	\includegraphics[width=4in]{identity.png}
    \caption*{Not sure why I just taught everyone to flawlessly impersonate me to pretty much anyone I know. Just remember to constantly bring up how cool it is that birds are dinosaurs and you'll be set.}
\end{figure}
\end{document}
