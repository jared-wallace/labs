\documentclass[letterpaper,12pt]{article}
\usepackage[utf8]{inputenc}
\usepackage{fullpage}
\usepackage{listings}
\usepackage{graphicx}
\lstset {
	language=[GNU]C++,
	showstringspaces=false,
	tabsize=2,
	basicstyle=\scriptsize,
        aboveskip={1.5\baselineskip},
        columns=fixed,
        showstringspaces=false,
        extendedchars=false,
        breaklines=true,
        prebreak = \raisebox{0ex}[0ex][0ex]{\ensuremath{\hookleftarrow}},
        frame=single,
        numbers=left,
        showtabs=false,
        showspaces=false,
        showstringspaces=false,
        identifierstyle=\ttfamily,
}	

\title{\Large CS 1428\\Lab 1 Sections 19 and 06} 
\author{Jared Wallace}
\date{}

\begin{document}

\maketitle

\section*{Topic}
Today we're covering file input and output.  The two objects we use
for file input and output are \textbf{ifstream} (for input files) and 
\textbf{ofstream} (for output files). For either object you must open 
a file before it can be used.  After opening a file you \textbf{MUST} 
check that the file was opened succesfully.  If you do not check for a
successful open and the open did fail then your program will most 
likely not crash.  Instead the program will run but behave strangely. 
We read in from and write out to our files using the insertion 
(\textbf{\textless\textless}) and extraction (\textbf{\textgreater\textgreater}) operators just like we do 
with \textbf{cin} and \textbf{cout}.

For example:

\begin{lstlisting}
int myInt;
ifstream fin;

fin.open("myfile.txt");
if (!fin) 
{
   cout << "Failed to open the file.\n":
   return 0;
}

fin >> myInt;
cout << "The integer is " << myInt << endl;
\end{lstlisting}

\section*{Part 1: Getting input from a file}
The provided input file \emph{input.txt} contains 3 grades each for 3
people for a total of 9 grades.  I want you to read in the grades for 
each person, calculate the average for each person and calculate the 
average for the whole "class".  After these calculations are complete 
I want you to print the results to the screen in a format that is 
readable.

I have provided you with a starter file \emph{input.cpp} which
contains the basic structure for your program and includes all
the libraries that you should need. Rename your copy \emph{input\_(netid).cpp}.

\newpage
\section*{Part 2: Printing output to a file}
Create your own program named \emph{output\_(netid).cpp} which will send
its output to the file \emph{output.txt}. I would like you to print
out several things.
\begin{itemize}
   \item Your name
   \item The lab section (C.S.1428.L0\#)
   \item Todays date
   \item Today's lab number
   \item A description of what we learned today.
\end{itemize}

The output should be neatly formatted, labeled and clear enough
that any random person reading it can tell what you're trying to
say.

When you are finished you must upload both source files (both input\_(netid).cpp and output\_(netid).cpp)
through the homework upload utility and print out a copy of each.
Staple this page to the front of those as a cover sheet, write your
name at the top and place it on my desk before leaving.
\section*{}
\includegraphics[width=7in]{cant_sleep.png}

\end{document}

