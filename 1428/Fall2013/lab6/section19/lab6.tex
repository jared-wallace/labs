\documentclass[letterpaper,12pt]{article}
\usepackage[utf8]{inputenc}
\usepackage{fullpage}
\usepackage{courier}
\usepackage[margin=0.75in]{geometry}
\usepackage{listings}
\usepackage{color}
\usepackage{graphicx}
\usepackage[width=4in]{caption}
\usepackage{hyphenat}

% Format a sectionless paragraph
\newcommand*\unparagraph{
	\par
	\nopagebreak
	\vskip3.25ex plus1ex minus.2ex
	\noindent
}

% define extra colors
\definecolor{dkgreen}{rgb}{0,0.6,0}
\definecolor{purple}{RGB}{159,0,197}

% define the code listing format
\lstset{
	language=C++,
	basicstyle=\ttfamily,
	backgroundcolor=\color{white},
	showspaces=false,
	showstringspaces=false,
	%frame=single,
	tabsize=3,
	keywordstyle=\color{purple},
	commentstyle=\color{dkgreen},
	stringstyle=\color{blue},
	escapeinside={\%*}{*)}
}

% define the title/header
\title{\Large CS 1428\\Lab 6 Section 19} 
\author{Jared Wallace}
\date{}

\begin{document}

\maketitle

\section*{Topic}
Today we are discussing the \lstinline$while$ loop and the \lstinline$do while$ loop. These two loops have very similar 
behavior with the only difference being that the \lstinline$do while$ loop will always execute once before
checking its condition.

% code example here
\unparagraph{}
\begin{lstlisting}[basicstyle=\footnotesize\ttfamily]
/*
 * Two different examples using while loops to check user input.
 * Each is subtly different
 */


 // First example
cout << "Please pick a number between 1 and 10: ";
cin >> theNumber;

while (theNumber < 1 || 10 < theNumber) {
	cout << theNumber << " is not between 1 and 10, try again.\n";
	cout << "Please pick a number between 1 and 10: ";
	cin >> theNumber;
}

// Second example
do {
	cout << "Please enter a number between 1 and 10: ";
	cin >> theNumber;
	cout << endl;
} while (theNumber < 1 || 10 < theNumber);

\end{lstlisting}

\newpage

\section*{Programming Task}
You will write a program that prompts the user to enter several grades.  As long
as the grades entered are valid (0 $\leq$ grade $\leq$ 100) the user will be able to continue
entering more grades. Once grade entry is complete the average should be calculated
and displayed.  Then the user is offered a choice to enter another set of grades 
or exit the program.

You will probably want two loops, one to handle the overall program, and one to handle 
getting grades. It will be necessary to keep track of the number of grades entered as well.


Be sure to prompt the user where appropriate, always assume the user does not know
the purpose of the program.

\vspace{15mm}

% Don't forget the submission instructions!
\unparagraph{} \textbf{Print out your source code and staple to the back of this page.  Also
upload your source file through the homework upload utility.}

\vspace{45mm}

% Comic at the bottom
\begin{figure}[ht!]
	\centering
	\includegraphics[width=6in]{new_car.png}
\end{figure}

\end{document}
